\documentclass[a4paper,10pt]{article}
%usepackgae
\usepackage[utf8]{vietnam}
\usepackage{amssymb}
\usepackage{amsmath}
\usepackage{amsfonts}

%title
\title{\vspace{-8ex} \bfseries ĐẠO HÀM}
\date{\vspace{-8ex}}

\begin{document}
\maketitle
Bài toán tìm đường tiếp tuyến của một đường cong hay bài toán tìm tốc độ tức thời của chuyển động đều liên quan đến cùng một loại phép tính Giới hạn, là phép tính Giới hạn của hình mẫu toán học \footnote{Rate of change (ROC): tỷ lệ thay đổi (trung bình hoặc tức thời) hay tốc độ biến thiên (trung bình hoặc tức thời)}Rate Of Change của hàm số $y=f(x)$ trong đoạn [$x$, $x_0$] có dạng ROC $= \frac{\Delta y}{\Delta x}=\frac{f(x)-f(x_0)}{x-x_0}$. Như vậy, phép tính Giới hạn của ROC được viết:


$$ \lim_{x \to x_0} \frac{f(x)-f(x_0)}{x-x_0} $$\\

Trên thực tế, hình mẫu này của phép tính Giới hạn phát sinh bất cứ khi nào chúng ta tính toán tốc độ biến thiên trong bất kỳ ngành Khoa học kỹ thuật nào, chẳng hạn như tốc độ phản ứng trong Hóa học hoặc chi phí cận biên trong Kinh Tế học. Hình mẫu Giới hạn này có tính phổ biến rất cao, vì vậy nó được gọi bằng một cái tên đặc biệt khác: Đạo hàm (Derivatives).\\

Trước khi phép tính Giới hạn ra đời thì Đạo hàm được định nghĩa trên ý tưởng Vi phân của Leibniz-Newton “và” kí hiệu của Largrange, khi đó:\\

Đạo hàm của số $y=f(x)$ tại $x_0$ được viết: $$f'(x_0)=\frac{dy}{dx}f(x_0)$$

Tổng quát cho mọi điểm: $$f'(x)=\frac{dy}{dx}f(x)$$

Sau khi phép tính Giới hạn ra đời, Đạo hàm mới được định nghĩa trên “phép tính Giới hạn” thành "một đối tượng toán học cơ bản".\\

Đạo hàm của số $y=f(x)$ tại $x_0$ được viết:
$$f'(x_0)=\lim_{x \to x_0} \frac{f(x)-f(x_0)}{x-x_0}$$

Tổng quát cho mọi điểm: $$ f'(x)=\lim_{\Delta x \to 0}\frac{\Delta x}{\Delta y} $$
\end{document}